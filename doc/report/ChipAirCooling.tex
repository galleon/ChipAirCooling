\documentclass[12pt, fleqn]{article} 
% Do not declare as a4 - see doclayout.sty
% To produce a4 version type: latex "\let\afour\def \input foo.tex"

\input{.latex}

% Nabla's specific macro packages
% ~~~~~~~~~~~~~~~~~~~~~~~~~~~~~~~
\usepackage{\macroPath/doclayout}           % Layout of the document
\usepackage{\wikkiMacroPath/reportlayout}   % Layout for a report
\usepackage{\macroPath/figtabeqnindbib}     % Bundled packages + some defined 
\usepackage{\macroPath/specialsymbolsfonts} % Special symbols esp for html
\usepackage{\macroPath/keywordstyles}       % Styles/indexing for types eg file
\usepackage{\macroPath/environmentformats}  % Environments esp. verbatim
\usepackage{\macroPath/productsversions}    % Product/company names/version nos.
\usepackage{\macroPath/general}             % fig/tab/eqn reference - TO GO
\usepackage{\macroPath/tensorCommon}        % Common tensor symbols
\usepackage{\macroPath/tensorEquation}      % Common tensor equations
\usepackage{\macroPath/tensorOperator}      % Common tensor operators
\usepackage{\macroPath/finiteVolume}        % Finite volume
\usepackage{\macroPath/hyperrefstyles}      % Hypertext stuff - must be last 

%%% Local Variables: 
%%% mode: latex
%%% TeX-master: t
%%% End: 


% Print DRAFT on every page
% ~~~~~~~~~~~~~~~~~~~~~~~~~
%\usepackage[light]{draftcopy}
%\draftcopySetGrey{0.9}
%\psdraft
%
\renewcommand{\ReportType}{Project Report}
\renewcommand{\ReportNo}{MS/EADS/001}

%
\begin{document}
%
%
\title
{
  \bfseries Simulation of the Air Cooling of Electronic Chip Assemblies 
}
%
\author
{
     \ReportType\\ \\ \\
     Wikki GmbH\\ \\
     \ReportNo
}
%
\maketitle
%
\section*{Abstract}
  This Report gives an overview of the governing equations for 3D
  buoyant flows with radiation and conjugate heat transfer, boundary
  conditions, software organization, case setup and sample results for
  the simulation environment developed under the {\bfseries Simulation
    of the Air Cooling of Electronic Chip Assemblies} project.

%
\section*{Introduction}

The chip cooling assembly is a complex device that is characterized by
a very complex physical phenomena that involves coupling of energy
fields between fluid and solid phase. Among the most important
phenomena, cooling of chip and motherboard assembly includes turbulent
buoyant flow with radiation in fluid phase, heat transfer with contact
resistance in solid domain with coupling between energy fields in
fluid and solid domains along the fluid-solid interface. In addition,
boundary conditions on inlet are time dependent and the volumetric
flow rate through the inlet boundary is controlled by a controller
that regulates the value of the volumetric flow rate depending in the
measured average temperature on the outlet. In order to simulate the
cooling of chip and motherboard assembly, governing partial
differential equations, material models and corresponding boundary
conditions are formulated and presented in this report together with
the solution algorithm and followed by exmaples of the usage of the
code.

\section*{Governing Equations}

The system of governing equations describing the buoyant turbulent
flow with conjugate heat transfer is given by equations of
conservation of mass, momentum and energy. Assuming that the
temperature variation is small among layers, Boussinesq approximation
can be used:

\begin{equation}\label{eq:boussinesq}
\rho = \rho_0(1 - \beta(T - T_0))
\end{equation}
Equation \ref{eq:boussinesq} is valid provided that the following
condition is satisfied:

\begin{equation}\label{eq:condition}
  \frac{\rho - \rho_0}{\rho_0} = \beta(T - T_0) \ll 1.
\end{equation}
Therefore, range of validity of Boussinesq approximation is for small
to moderate density variations. In that case transient terms in
governing equations can be neglected and density can be considered
constant except in buoyant terms. Furthermore, thermal heating and
viscous dissipation are considered small due to small density
variations. With these assumption the resulting system of governing
equations for the fluid phase is as follows:

Continuity equation:
\begin{equation}\label{eq:fluid-continuity}
  \frac{\partial u_i}{\partial x_i} = 0
\end{equation}

Momentum equation:
\begin{equation}\label{eq:fluid-momentum}
\frac{\partial u_i}{\partial t} +
  \frac{\partial(u_ju_i)}{\partial x_j} = 
    -\frac{\partial}{\partial x_i}
    \left( \frac{p}{\rho_0} \right) + g_i
  -\beta(T - T_0)g_{i} 
  + \frac{\partial}{\partial x_j}
  \left[\frac{\mu_{eff}}{\rho_0}
    \left(\frac{\partial u_i}{\partial x_j} + 
    \frac{\partial u_j}{\partial u_i}\right)\right]
\end{equation}

Energy equation:
\begin{equation}\label{eq:fluid-energy}
\frac{\partial T}{\partial t} +
\frac{\partial}{\partial x_i}(u_i T) = 
\frac{\partial}{\partial x_j}
\left(\frac{k_{eff}}{\rho_0 c_p}\frac{\partial T}{\partial x_j}\right) 
+ \frac{1}{\rho_0 c_p} S_r
\end{equation}

Symbol $g_i$ is the vector of gravitational acceleration, $S_r$ is the
contribution from the various sources including the radiation
contribution and all other symbols have their usual meanings.

The set of equations describes the heat transfer in the solid phase
and it is given by the following energy equation:
\begin{equation}\label{eq:solid-energy}
\rho C \frac{\partial T}{\partial t} +
\frac{\partial}{\partial x_i}\left(k \frac{\partial T}{\partial x_i}\right) 
+ S_s = 0
\end{equation}

In order to compute turbulent transport coefficients, $\mu_{eff}$ and
$k_{eff}$, turbulence equations must be added to the the set of
equations. Naturally, the exact form of the turbulent equations will
depend on the selected turbulence model. In this report, $k-\epsilon$
model with wall functions was used to compute results.

In addition, the conduction of the heat between chip and motherboard
assembly can be altered due to surface roughness and/or bonding. This
phenomena is is accounted for through a contact resistance model:
\begin{equation}
  q_i = - \beta_c \frac{\partial T}{\partial x_i}
\end{equation}
The coefficient $\beta_c$ represents thermal conductivity that depends
on the type of the contact between two solid materials and it is
usually determined experimentally. Contact resistance model is an
additional equation that is applied to a surface inside of the solid
computation domain.

\section*{Solution Algorithm}

Two solution algorithms for simulating the steady and unsteady flow in
chip cooling assembly were implemented.  The steady state solution of
the fluid flow is performed by the well known SIMPLE algorithm. The
solution algorithm is implemented in the file
``chipCoolingSimpleFoam.C'' and it uses the predictor-corrector scheme
to update pressure and velocity fields. The steady state code can be
found in the directory ``/src/chipAirCoolingSimpleFoam''. The unsteady
solution algorithm was implemented in the main code
``chipAirCoolingFoam.C''and it uses PISO time stepping algorithm for
time marching. The unsteady code can be found in the
``/src/chipAirCoolingFoam'' directory. Both solution algorithms are
controlled through ``controlDict'', ``fvSchemes'' and ``fvSolution''
files that define parameters of the discretization scheme, solution
algorithm and overall behavior.

The energy equation solution is complicated by the fact that the
temperature fields in solid and fluid regions are coupled through a
common boundaries. In order to solve the coupled problem, the coupled
matrix technology available in OpenFOAM is used to create a single
matrix for energy equation that spans both fluid and solid
regions. Once the velocity field is available by solving the fluid
equations, coupled matrix for energy equation for both fluid and solid
phases is formed and solved. The code for coupling of energy equations
in fluid and solid phase can be found in ``solveEnergy.H'' file. The
main point of the coupling code is a creation of the coupled matrix
that contains temperature fields for both fluid and solid phase:

\begin{verbQuoteSmall}
fvScalarMatrix* TFluid = new fvScalarMatrix
(
    rho*Cp*
    (
        fvm::ddt(T)
      + fvm::div(phi, T)
    )
  - fvm::laplacian(kappaEff, T)
    ==
    radiation->Ru()
  - fvm::Sp(4.0*radiation->Rp()*pow3(T), T)
);

TFluid->boundaryManipulate(T.boundaryField());
TFluid->relax(mesh.relaxationFactor("T+T"));

fvScalarMatrix* TSolid = new fvScalarMatrix
(
    fvm::ddt(rhoCpsolid, Tsolid)
  - fvm::laplacian(ksolidf, Tsolid, "laplacian(k,T)")
  + fvm::SuSp(-solidThermo.S()/Tsolid, Tsolid)
);

TSolid->relax(mesh.relaxationFactor("T+T"));

// Add fluid equation
TEqns.set(0, TFluid);

// Add solid equation
TEqns.set(1, TSolid);

TEqns.solve();
\end{verbQuoteSmall}

\section*{Software Organization}

Software implementation of the chip air cooling simulation tool
consists of the solver executable and multi-material library
components. Solver executable is capable of simulating of complex
physical phenomena including buoyant flow with heat and radiative
transfer in fluid and heat transfer and contact resistance in solid
domain. Multi-material library is capable of providing the material
properties and local resistance laws that are needed to solve coupled
heat transfer problem.

The steady main code can be found in ``src/chipAirCoolingSimpleFoam'' directory
and the main code in the file ``chipAirCoolingSimpleFoam.C''. Multi-material
library can be found in ``src/airbusMaterialModels'' directory while tutorial cases
can be found in ``run/''. The unsteady main code can be found in
``src/chipAirCoolingFoam'' in the file ``chipAirCoolingSimpleFoam.C''.


\subsection*{Tutorials}
Tutorial1 example in in the ``run'' directory represents a simple
example of a steady state conjugate heat transfer in lid driven
cavity. Conjugate heat transfer is taking the place through a shared
boundary between fluid and solid domains. We first need to generate
computation mesh for both solid and fluid regions. In the directory
``tutorial1'', solid phase mesh can be found under
``tutorial1/heatedContactBlock''. In order to facilitate a better
understanding of the mesh setup, directory ``constant/polyMesh''
contains pre-generated mesh. It is important to observe the following
entry in the file\\
``tutorial1/heatedContactBlock/constant/polyMesh/boundary'':

\begin{verbQuoteSmall}
left
{
    type regionCouple;
    nFaces 10;
    startFace 200;

    shadowRegion    region0;
    shadowPatch     right;
    attached        on;
    attachedWalls   on;
}
\end{verbQuoteSmall}

The patch called ``left'' is the solid side of the coupling patch and
as such must contain the entries describing the fluid side of the
corresponding patch. Similarly, on the fluid side the corresponding
patch is called ``right'' and it has the following entries in the
``tutorial1/conjugateContactCavity/constant/boundary'' file:

\begin{verbQuoteSmall}
right
{
    type regionCouple;
    nFaces 10;
    startFace 200;

    shadowRegion    solid;
    shadowPatch     left;
    attached        on;
    attachedWalls   on;
}
\end{verbQuoteSmall}

In order to setup the solid materials and contact resistance
properties, cell and face zones have to be created by executing the
following commands in setSet utility:

\begin{verbQuoteSmall}
setSet

cellSet solidBlock new boxToCell (0.15 0 0) (0.2 0.1 0.01);
faceSet contactSurface new boxToFace (0.1499 0 0) (0.1501 0.1 0.01);
quit

setsToZones -noFlipMap
\end{verbQuoteSmall}

The names ``solidBlock'' and ``contactSurface'' can now be used to
specify material and contact properties in
``tutorial1/heatedContactBlock/constant/thermalProperties'' file:

\begin{verbQuoteSmall}
thermal
\{
    type                    multiMaterialZones;

    laws
    (
        material0
        \{
            type            constant;
            rho             rho [1 -3 0 0 0] 10;
            C               C [0 2 -2 -1 0] 1000;
            k               k [1 1 -3 -1  0] 0.001;

            zones           ( solidBlock );
        \}
    );

    gaps
    (
        air // gap 0
        \{
            type            constant;
            beta            beta [1 1 -3 -1 0] 0.00025;
            zones           ( contactSurface );
        \}
    );
\}
\end{verbQuoteSmall}

In order to setup the fluid phase for coupled problem solution,
several symbolic links from the solid directory are required:

\begin{itemize}
\item In the directory ``tutorial1/conjugateContactCavity/0'' the
  following symbolic link is required to point to the corresponding
  ``0/'' directory on the solid side: `` solid $\rightarrow$
  ../../heatedContactBlock/0''
\item In the directory ``tutorial1/conjugateContactCavity/constant''
  the following link must point to the corresponding ``constant/''
  directory on the solid side: `` solid $\rightarrow$
  ../../heatedContactBlock/constant''
\item In the directory
  ``tutorial1/conjugateContactCavity/constant/polyMesh/'' the
  following symbolic links are required: \\ `` cellZones $\rightarrow$
  ../../../heatedContactBlock/constant/polyMesh/cellZones'',
  \\ ``faceZones $\rightarrow$
  ../../../heatedContactBlock/constant/polyMesh/faceZones'', \\ ``sets
  $\rightarrow$ ../../../heatedContactBlock/constant/polyMesh/sets'',
  and\\ ``solid $\rightarrow$
  ../../../heatedContactBlock/constant/polyMesh
\end{itemize}

Physical and model constants together with the flow solver control
files can be found in the ``run/tutorial1/constant''. A sample results
of runnning the code can be seen in figures \ref{fig:fig1}
through \ref{fig:fig3}. Figures \ref{fig:fig1} and \ref{fig:fig2} show
the initial and the final temperature distributions where the effect
of the contact resistance can be clearly seen in the solid
region. Figure \ref{fig:fig3} shows the distribution of the velocity
in the fluid domain clearly indicating that there is no velocity field
in the solid region.

\begin{figure*}[htbp]
\begin{center}
\epsfig{figure=figures/Tinit.eps, width = 10.0 cm,angle=0,clip=}
\caption{Initial temperature distribution}
\label{fig:fig1}
\end{center}
\end{figure*}

\begin{figure*}[htbp]
\begin{center}
\epsfig{figure=figures/Tfinal.eps, width = 10.0 cm,angle=0,clip=}
\caption{Final temperature distribution}
\label{fig:fig2}
\end{center}
\end{figure*}

\begin{figure*}[htbp]
\begin{center}
\epsfig{figure=figures/U.eps, width = 10.0 cm,angle=0,clip=}
\caption{velocity magnitude distribution}
\label{fig:fig3}
\end{center}
\end{figure*}


Tutorial2 represents a chip-motherboard assembly consisting of solid
and fluid domains. The tutorial2 files can be found in
``run/tutorial2'' directory. Overall procedure for
creating the mesh is exactly the same as in tutorial1 example but this
time executed in the directory of the tutorial2 example. Tutorial2
represents a steady state solver that can be ran with
chipAirCoolingSimpleFoam executable in order to produce results.

A sample results of running the executable in tutorial2 is shown in
figures \ref{fig:fig4} and \ref{fig:fig5}. Figure \ref{fig:fig4} show
the temperature distribution in the fluid and solid regions whereas
figure \ref{fig:fig5} shows the velocity magnitude field in the fluid
region.the effect of the contact resistance and the radiative source
terms can be clearly seen in those figures.

\begin{figure*}[htbp]
\begin{center}
\epsfig{figure=figures/Ttut2.eps, width = 10.0 cm,angle=0,clip=}
\caption{Temperature distribution}
\label{fig:fig4}
\end{center}
\end{figure*}

\begin{figure*}[htbp]
\begin{center}
\epsfig{figure=figures/Utut3.eps, width = 10.0 cm,angle=0,clip=}
\caption{velocity magnitude distribution}
\label{fig:fig5}
\end{center}
\end{figure*}


Tutorial3 represents the same chip-motherboard assembly with the
difference that this example is unsteady flow and heat transfer with a
proportional controller that regulates the variables on inflow
boundary based on the observations made on the outflow boundary. The
tutorial3 example can be found in ``run/tutorial3''
directory and the results can be produced by running the
``chipAirCoolingFoam'' executable. 

Boundary conditions that define the inlet boundary variables and
proportional controller parameters are given through the following
velocity entry example:

\begin{verbQuoteSmall}
    fluidInlet
    \{
        type            controlledParabolicVelocity;
        Umean           5;
        n               (0 1 0);
        y               (1 0 0);

        obsFieldName    T;
        obsPatchName    fluidOutlet;
        target          290;
        gain            1e-6;

        value           uniform (0 5 0);
    \}
\end{verbQuoteSmall}

The entry ``obsFieldName'' defines the field that is being observed on
the patch which name is defined in the entry ``obsPatchName''. The
target value of the observed field is given through entry ``target''
whereas the system gain used in proportional controller is given
through the entry ``gain''. The rest of the inputs in the velocity
field boundary entry defines the parabolic velocity
profile. Controller entries are used by the proportional controller
code in order to drive the inlet control parameter to the value that
gives desired observed variable value on the observed patch. The
proportional controller achieves the target value by adjusting the
control parameter through the following relation:

\begin{equation}
  \Phi_c = G(\Psi_o - \Psi_t)
\end{equation}
Here $\Phi_c$ is a control parameter, $\Psi_o$ and $\Psi_t$ are
observed and target variables, whereas $G$ is the system gain. In the
example given in tutorial3, observed variable is averaged temperature
field T on the outlet, control variable is the volumetric flow
rate. The target value was selected to be $T= 290^o\,K$ and the gain
of the control system is selected to be $G=1E-06$ so that the
controller does not cause excessive flow rates on the inlet. The
controller will achieve the given value in the time marching algorithm
thus allowing to observe time dependence of the flow and temperature
fields.

A sample results of running the executable code with the proportional
control are given in figures \ref{fig:fig6} and \ref{fig:fig7}. These
results show the effect of the proportional controller on both velocity
and temperature fields when compared to figures \ref{fig:fig4}
and \ref{fig:fig5}. Proportional controller is matching closely the
desired temperature profile on the outlet causing somewhat different
field distributions compared to the case without control.

\begin{figure*}[htbp]
\begin{center}
\epsfig{figure=figures/Ttut3.eps, width = 10.0 cm,angle=0,clip=}
\caption{Temperature distribution}
\label{fig:fig6}
\end{center}
\end{figure*}

\begin{figure*}[htbp]
\begin{center}
\epsfig{figure=figures/Utut3.eps, width = 10.0 cm,angle=0,clip=}
\caption{Velocity magnitude distribution}
\label{fig:fig7}
\end{center}
\end{figure*}



\section*{Summary}

In the course of the project, steady and unsteady flow solvers capable
of simulating buoyant radiative turbulent flow with conjugate heat
transfer was developed. In addition to the solvers developed, material
library that defines zonal models for the solid material properties,
contact resistance and heat equation source terms was
developed. Furthermore, parabolic velocity profile with the simple
proportional controller was also developed to allow the simulation of
the controlled cooling of the chip-motherboard assemblies.


\end{document}

